\documentclass[10pt]{article}
\usepackage[utf8x]{inputenc}
\usepackage{polski}

\usepackage{amsmath}
\usepackage{amsthm}
\usepackage{amssymb}

\usepackage{graphics}
\usepackage{pdfpages}
\DeclareGraphicsRule{.1}{mps}{*}{}
\usepackage{epstopdf}

% \newcommand{\hint}[2]{\begin{flushleft}\textbf{\ref{#1}}\quad #2\end{flushleft}}

\usepackage{geometry}
\geometry{a4paper, margin=2cm}

\begin{document}
    
%NAGLOWEK

    \noindent
    \begin{minipage}{0.5\textwidth}
        \LARGE{\textsf{\textbf{Zadanie: MYS\\Myszy}}}
    \end{minipage}
    \begin{minipage}{0.5\textwidth}
        \begin{flushright}
            \includegraphics[height=1.5cm]{logo.png}
        \end{flushright}
    \end{minipage}
    
    \noindent\rule{\textwidth}{0.4pt}
    
    \noindent\textbf{Akademia Programowania PWSW, dzień ?, Dostępna pamięć: 128 MB.}
    \vspace{1em}
    
%TRESC
    
    \noindent
    Ponownie odwiedzamy kota Bitoma, tym razem spotykając go polującego wraz z przyjaciółmi w pewnym długim korytarzu podzielonym na $n$ fragmentów (od 1 do $n$). Na każdym z fragmentów znajduje się pewna liczba myszy (być może zero). Każdy z $k$ kotów będzie polował na pewnym spójnym obszarze korytarza, przy czym obszary te nie będą na siebie zachodziły, aby koty nie pokłóciły się o zdobycze. Kot polujący na przedziale $[i, j]$ $(1\leq i\leq j\leq n)$ jest w stanie złapać \texttt{max($s - (j - i)^{2}$, $0$)} myszy, gdzie $s$ to łączna liczba myszy przesiadująca na fragmencie $[i, j]$ korytarza.
    
    Pomóż kotom wybrać najlepsze miejsca do polowania i określ ile myszy są w stanie upolować.
    

%WEJSCIE

    \section*{Wejście}
    
    W pierwszym wierszu standardowego wejścia znajdują się dwie liczby całkowite $n$ i $k$, oznaczające odpowiednio liczbę fragmentów, na które podzielony jest korytarz oraz liczbę polujących kotów. W kolejnym wierszu znajduje się ciąg $n$ liczb całkowitych z przedziału od 0 do $10^{6}$, oznaczający ile myszy znajduje się na każdym z odcinków korytarza.

%WYJSCIE

    \section*{Wyjście}
    
    Na standardowe wyjście należy wypisać jedną liczbę całkowitą, oznaczającą maksymalną liczbę myszy jakie mogą zostać złapane przez koty.

%PRZYKLAD

    \section*{Przykład}
    
    \noindent
    \begin{minipage}[t]{0.5\textwidth}
        Dla danych wejściowych:\vspace{1ex}\\
        \texttt{8 2\\1 5 1 4 3 2 7 0}
    \end{minipage}
    \begin{minipage}[t]{0.5\textwidth}
        poprawnym wynikiem jest:\vspace{1ex}\\
        \texttt{14}
    \end{minipage}
    
    \vspace{2ex}
    \noindent\textbf{Wyjaśnienie przykładu:} Pierwszy kot poluje na odcinku od 2 do 4 łapiąc 6 z 10 myszy, zaś drugi kot poluje na odcinku od 5 do 7 łapiąc 8 z 12 myszy.
    
%OCENIANIE

    \section*{Ocenianie}
        
    Zestaw testów dzieli się na następujące podzadania. Testy do każdego podzadania składają się z jednej lub większej liczby osobnych grup testów.
    
    \begin{center}
        \begin{tabular}{ |c|p{9cm}|c| }
            \hline
            \textbf{Podzadanie} & \textbf{Warunki} & \textbf{Liczba punktów}\\
            \hline
            1 & $1 \leq k \leq n \leq 10$ & 15\\
            \hline
            2 & $1 \leq k \leq n \leq 200$ & 30\\
            \hline
            3 & $1 \leq k \leq n \leq 2000$ & 55\\
            \hline
        \end{tabular}
    \end{center}

\end{document}
\documentclass[10pt]{article}
\usepackage[utf8x]{inputenc}
\usepackage{polski}

