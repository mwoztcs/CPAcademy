\documentclass[10pt]{article}
\usepackage[utf8x]{inputenc}
\usepackage{polski}

\usepackage{amsmath}
\usepackage{amsthm}
\usepackage{amssymb}

\usepackage{graphics}
\usepackage{pdfpages}
\DeclareGraphicsRule{.1}{mps}{*}{}
\usepackage{epstopdf}

% \newcommand{\hint}[2]{\begin{flushleft}\textbf{\ref{#1}}\quad #2\end{flushleft}}

\usepackage{geometry}
\geometry{a4paper, margin=2cm}

\begin{document}
    
%NAGLOWEK

    \noindent
    \begin{minipage}{0.5\textwidth}
        \LARGE{\textsf{\textbf{Zadanie: LAM\\Lampki}}}
    \end{minipage}
    \begin{minipage}{0.5\textwidth}
        \begin{flushright}
            \includegraphics[height=1.5cm]{logo.png}
        \end{flushright}
    \end{minipage}
    
    \noindent\rule{\textwidth}{0.4pt}
    
    \noindent\textbf{Akademia Programowania PWSW, dzień ?, Dostępna pamięć: 128 MB.}
    \vspace{1em}
    
%TRESC
    
    \noindent
    Święta zbliżają się wielkimi krokami, dlatego Bajtek postanowił udać się do sklepu i wybrać lampki na choinkę. Wśród wielu wzorów i kolorów najbardziej urzekły go \textit{Świetliste pasy}, gdzie każda z lampek w łańcuchu świeci na jeden z $m$ kolorów oznaczonych od 1 do $m$, a dzięki specjalnemu przełącznikowi (działającemu również w tym zakresie wartości) Bajtek może zapalić jedynie te lampki, których oznaczenia kolorów należą do pewnego spójnego przedziału $[l, r]$, dla $l \leq r$. W ten sposób świecące lampki tworzą \textit{świetliste grupy} oddzielone między sobą co najmniej jedną zgaszoną lampką.

    Napisz program, który powie Bajtkowi, ile łącznie świtlistych grup są w stanie wygenerować wybrane przez niego lampki, dla wszystkich możliwych przedziałów ustawionych na specjalnym przełączniku.

%WEJSCIE

    \section*{Wejście}
    
    W pierwszym wierszu standardowego wejścia znajdują się dwie liczby całkowite $n$ i $m$, oznaczające odpowiednio liczbę światełek w łańcuchu oraz liczbę kolorów. W drugim wierszu znajduje się ciąg $n$ liczb całkowitych z przedziału od 1 do $m$, oznaczających kolejne kolory lampek w wybranym przez Bajtka łańcuchu.

%WYJSCIE

    \section*{Wyjście}
    
    Na standardowe wyjście należy wypisać jedną liczbę całkowitą, oznaczającą liczbę wszystkich świetlistych grup, jakie Bajtek jest w stanie uzyskać dla różnych przedziałów ustawionych na specjalnym przełączniku.  

%PRZYKLAD

    \section*{Przykład}
    
    \noindent
    \begin{minipage}[t]{0.5\textwidth}
        Dla danych wejściowych:\vspace{1ex}\\
        \texttt{3 3\\2 1 3}
    \end{minipage}
    \begin{minipage}[t]{0.5\textwidth}
        poprawnym wynikiem jest:\vspace{1ex}\\
        \texttt{7}
    \end{minipage}
    
    \vspace{2ex}
    \noindent\textbf{Wyjaśnienie przykładu:} Liczba świetlistych grup dla poszczególnych przedziałów wynosi: [1, 1] = 1; [1, 2] = 1; [1, 3] = 1; [2, 2] = 1; [2, 3] = 2; [3, 3] = 1. 
    
%OCENIANIE

    \section*{Ocenianie}
        
    Zestaw testów dzieli się na następujące podzadania. Testy do każdego podzadania składają się z jednej lub większej liczby osobnych grup testów.
    
    \begin{center}
        \begin{tabular}{ |c|p{9cm}|c| }
            \hline
            \textbf{Podzadanie} & \textbf{Warunki} & \textbf{Liczba punktów}\\
            \hline
            1 & $1 \leq n, m \leq 100$ & 10\\
            \hline
            2 & $1 \leq n, m \leq 2000$ & 30\\
            \hline
            3 & $1 \leq n, m \leq 200000$ & 60\\
            \hline
            
        \end{tabular}
    \end{center}

\end{document}
