\documentclass[10pt]{article}
\usepackage[utf8x]{inputenc}
\usepackage{polski}

\usepackage{amsmath}
\usepackage{amsthm}
\usepackage{amssymb}

\usepackage{graphics}
\usepackage{pdfpages}
\DeclareGraphicsRule{.1}{mps}{*}{}
\usepackage{epstopdf}

\usepackage{geometry}
\geometry{a4paper, margin=2cm}

\begin{document}
    
%NAGLOWEK

    \noindent
    \begin{minipage}{0.5\textwidth}
        \LARGE{\textsf{\textbf{Zadanie: VIA\\Via ferrata}}}
    \end{minipage}
    \begin{minipage}{0.5\textwidth}
        \begin{flushright}
            \includegraphics[height=1.5cm]{logo.jpg}
        \end{flushright}
    \end{minipage}
    
    \noindent\rule{\textwidth}{0.4pt}
    
    \noindent\textbf{Akademia Programowania PWSW, dzień I, Dostępna pamięć: 128 MB.}
    \vspace{1em}
    
%TRESC
    
    \noindent
    W Bajmitach - największym paśmie górskim Bajtocji każdy wspinacz znajdzie coś dla siebie. Jednak spośród wszystkich dostępnych via ferrat\footnote{,,żelazna droga" - szlak turystyczny o charakterze wspinaczkowym, wyposażony dla celów autoasekuracji w stalową linę.} najbardziej obleganą jest \textit{Sci Club 18}\footnote{rzeczywista nazwa trudnej via ferraty znajdującej się w Dolomitach}, która podzielona jest na $n$ mniejszych odcinków, każdy o pewnej trudności. 

    Z powyższej drogi zamierza skorzystać $m$ śmiałków, każdy o pewnym stopniu zaawansowania. Aby wspinacz był w stanie pokonać kolejny odcinek trasy jego doświadczenie musi być wyższe od trudności drogi. W przeciwnym razie wspinacz zatrzymuje się w danym miejscu na skalnej półce - dalej nie będzie mógł już iść. Jeśli pewna osoba zatrzyma się, to każda kolejna nie będzie mogła wejść wyżej - będzie musiała zatrzymać się na poprzedniej skalnej półce (kolejne fragmenty via ferraty zakończone są skalnymi półkami) w związku z ograniczonym miejscem.

    Napisz program, który pomoże stwierdzić, gdzie zatrzymali się śmiałkowie.

%WEJSCIE

    \section*{Wejście}
    
    W pierwszym wierszu standardowego wejścia znajdują się dwie liczby całkowite $n$ i $m$ $(1 \leq n, m \leq 200000)$, oznaczające liczbę odcinków via ferraty i liczbę śmiałków. W drugim wierszu znajduje się ciąg $n$ liczb całkowitych $d_{1}, d_{2}, \ldots, d_{n}$ $(1 \leq d_{i} \leq 10^{9})$, gdzie $d_{i}$ oznacza stopień trudności $i$-tego odcinka drogi. W trzecim wierszu znajduje się $m$ liczb całkowitych $w_{1}, w_{2}, \ldots, w_{m}$ $(1 \leq w_{i} \leq 10^{9})$, gdzie $w_{i}$ oznacza poziom zaawansowania $i$-tego wspinacza.

%WYJSCIE

    \section*{Wyjście}
    
    Na standardowe wyjście należy wypisać jedną linię zawierającą $m$ liczb, oznaczających maksymalny numer skalnej półki jaki jest w stanie osiągnąć $i$-ty wspinacz, w kolejności podanej na wejściu.

%PRZYKLAD

    \section*{Przykład}
    
    \noindent
    \begin{minipage}[t]{0.5\textwidth}
        Dla danych wejściowych:\vspace{1ex}\\
        \texttt{4 5\\2 5 5 1\\7 6 5 4 3}
    \end{minipage}
    \begin{minipage}[t]{0.5\textwidth}
        poprawnym wynikiem jest:\vspace{1ex}\\
        \texttt{4 3 1 0 0}
    \end{minipage}
    
    \vspace{2ex}
    \noindent\textbf{Wyjaśnienie przykładu:} Pierwszy wspinacz jest w stanie przejść całą trasę, aż do ostatniej skalnej półki. Drugi zatrzyma się na wcześniejszej półce, ponieważ ostatnią zajął pierwszy wspinacz. Trzeci ze śmiałków zatrzyma się przed odcinkiem o trudności 5, zaś każdy następny ze względu na zablokowaną skalną półkę nie będzie w stanie rozpocząć wspinaczki.
    
%OCENIANIE

    \section*{Ocenianie}
        
    Zestaw testów dzieli się na następujące podzadania. Testy do każdego podzadania składają się z jednej lub większej liczby osobnych grup testów.
    
    \begin{center}
        \begin{tabular}{ |c|p{9cm}|c| }
            \hline
            \textbf{Podzadanie} & \textbf{Warunki} & \textbf{Liczba punktów}\\
            \hline
            1 & $1 \leq n, m \leq 1000$ & 25\\
            \hline
            2 & $w_{i}\geq w_{j}$, dla każdej pary $i$, $j$, gdzie $i < j$ & 25\\
            \hline
            3 & brak dodatkowych ograniczeń & 50\\
            \hline
        \end{tabular}
    \end{center}

\end{document}
