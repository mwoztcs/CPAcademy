\documentclass[10pt]{article}
\usepackage[utf8x]{inputenc}
\usepackage{polski}

\usepackage{amsmath}
\usepackage{amsthm}
\usepackage{amssymb}

\usepackage{graphics}
\usepackage{pdfpages}
\DeclareGraphicsRule{.1}{mps}{*}{}
\usepackage{epstopdf}

\usepackage{geometry}
\geometry{a4paper, margin=2cm}

\begin{document}
    
%NAGLOWEK

    \noindent
    \begin{minipage}{0.5\textwidth}
        \LARGE{\textsf{\textbf{Zadanie: JAP\\Japoński ogród}}}
    \end{minipage}
    \begin{minipage}{0.5\textwidth}
        \begin{flushright}
            \includegraphics[height=1.5cm]{logo.jpg}
        \end{flushright}
    \end{minipage}
    
    \noindent\rule{\textwidth}{0.4pt}
    
    \noindent\textbf{Akademia Programowania PWSW, dzień II, Dostępna pamięć: 128 MB.}
    \vspace{1em}
    
%TRESC
    
    \noindent
    Zgodnie z życzeniem królowej część królewskich ogrodów zostanie przebudowana i urządzona na wzór pięknych japońskich ogrodów. W tym celu nadworny ogrodnik musi przedstawić królowej różne projekty, a zadanie to nie jest proste, ze względu na liczne zasady jakie musi spełniać taki ogród. 
    
    Ogród taki musi być zbudowany na planie trójkąta prostokątnego, a w jego wierzchołkach należy wybudować świątynie, które należy połączyć ścieżkami w linii prostej.
    
    Cały królewski ogród możne być reprezentowany jako krata liczb całkowitych. Stara świątynia stojąca w punkcie $(0, 0)$, zgodnie z życzeniem królowej, znajdzie się w jednym z wierzchołków trójkątnego ogrodu. Następna świątynia stanie w punkcie $(x, 0)$, gdzie $x$ jest pewną liczbą całkowitą. Ostatnia świątynia zostanie wybudowana w punkcie $(x, y)$ - na miejscu jednego z drzew rosnących w ogrodzie. Wszystkie trzy świątynie zostaną połączone ścieżkami (co może wiązać się z wycięciem drzew leżących na linii łączącej pewne dwie świątynie).
    
    Pomóż nadwornemu ogrodnikowi i wyznacz liczbę drzew jaka znajdzie się we wnętrzu ogrodu dla wszystkich możliwych położeń świątyń.

%WEJSCIE

    \section*{Wejście}
    
    W pierwszym wierszu standardowego wejścia znajduje się jedna liczba całkowita $n$ $(1\leq n\leq 200000)$, oznaczająca liczbę drzew w ogrodzie. W $n$ kolejnych wierszach znajdują się dwie liczby całkowite $x_{i}$ i $y_{i}$ $(1\leq x_{i}, y_{i}\leq 10^{6})$, oznaczające położenie $i$-tego drzewa w królewskim ogrodzie (z oczywistych względów w jednym punkcie znajduje się co najwyżej jedno drzewo).

%WYJSCIE

    \section*{Wyjście}
    
    Na standardowe wyjście należy wypisać $n$ wierszy, w każdym jedną liczbę całkowitą, oznaczającą liczbę drzew znajdującą się we wnętrzu ogrodu, jeżeli ostatnia świątynia powstanie na miejscu $i$-tego drzewa.

%PRZYKLAD

    \section*{Przykład}
    
    \noindent
    \begin{minipage}[t]{0.5\textwidth}
        Dla danych wejściowych:\vspace{1ex}\\
        \texttt{4\\2 1\\3 2\\4 3\\4 2}
    \end{minipage}
    \begin{minipage}[t]{0.5\textwidth}
        poprawnym wynikiem jest:\vspace{1ex}\\
        \texttt{0\\1\\2\\0}
    \end{minipage}
    
    \vspace{2ex}
    \noindent\textbf{Wyjaśnienie przykładu:} Pierwszy ogród: we wnętrzu nie znajduje się żadne drzewo. Drugi ogród: we wnętrzu znajduje się drzewo na pozycji $(2, 1)$. Trzeci ogród: we wnętrzu znajdują się dwa drzewa - $(2, 1)$ i $(3, 2)$. Drzewo $(4, 2)$ leży na ścieżce łączącej świątynie $(4, 3)$ i $(4, 0)$, dlatego zostanie wycięte i nie znajdzie się w ogrodzie. Czwarty ogród: we wnętrzu nie znajduje się żadne drzewo. Drzewo $(2, 1)$ będzie leżało na ścieżce łączącej świątynie $(4, 2)$ i $(0, 0)$, dlatego zosatnie wycięte i nie znajdzie się w ogrodzie.
    
%OCENIANIE

    \section*{Ocenianie}
        
    Zestaw testów dzieli się na następujące podzadania. Testy do każdego podzadania składają się z jednej lub większej liczby osobnych grup testów.
    
    \begin{center}
        \begin{tabular}{ |c|p{9cm}|c| }
            \hline
            \textbf{Podzadanie} & \textbf{Warunki} & \textbf{Liczba punktów}\\
            \hline
            1 & $1\leq n\leq 1000$ & 35\\
            \hline
            2 & wszystkie drzewa w ogrodzie leżą na jednej prostej & 20\\
            \hline
            3 & brak dodatkowych ograniczeń & 45\\
            \hline
        \end{tabular}
    \end{center}

\end{document}
\documentclass[10pt]{article}
\usepackage[utf8x]{inputenc}
\usepackage{polski}

