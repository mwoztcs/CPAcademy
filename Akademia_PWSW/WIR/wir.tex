\documentclass[10pt]{article}
\usepackage[utf8x]{inputenc}
\usepackage{polski}

\usepackage{amsmath}
\usepackage{amsthm}
\usepackage{amssymb}

\usepackage{graphics}
\usepackage{pdfpages}
\DeclareGraphicsRule{.1}{mps}{*}{}
\usepackage{epstopdf}

\usepackage{geometry}
\geometry{a4paper, margin=2cm}

\begin{document}
    
%NAGLOWEK

    \noindent
    \begin{minipage}{0.5\textwidth}
        \LARGE{\textsf{\textbf{Zadanie: WIR\\Wirus}}}
    \end{minipage}
    \begin{minipage}{0.5\textwidth}
        \begin{flushright}
            \includegraphics[height=1.5cm]{logo.jpg}
        \end{flushright}
    \end{minipage}
    
    \noindent\rule{\textwidth}{0.4pt}
    
    \noindent\textbf{Akademia Programowania PWSW, dzień II, Dostępna pamięć: 128 MB.}
    \vspace{1em}
    
%TRESC
    
    \noindent
    W bajtockiej serwerowni znajduje się $n$ serwerów (o identyfikatorach od 1 do $n$) połączonych między sobą w sieć tworzącą drzewo\footnote{graf spójny bez cykli.}. Z powodu błędu jednego z pracowników na serwer o numerze 1 dostał się wirus, który rozprzestrzenia się pomiędzy kolejnymi połączonymi serwerami. 

    Bitek musi jak najszybciej zareagować i odciąć połączenia pomiędzy niektórymi serwerami, aby ochronić odpowiednio dużo maszyn. Dla każdego połączenia pomiędzy serwerami, Bitek będący administratorem sieci wie, czy może odciąć to połączenie, czy nie. 
    
    Ponieważ liczy się czas, napisz program i pomóż Bitkowi określić jaka jest minimalna liczba połączeń, które Bitek musi odciąć, aby zainfekowanych pozostało co najwyżej $m$ serwerów.

%WEJSCIE

    \section*{Wejście}
    
    W pierwszym wierszu standardowego wejścia znajdują się dwie liczby całkowite $n$ i $m$ $(1\leq m\leq n\leq 200000)$, oznaczające liczbę serwerów i maksymalną liczbę serwerów jaka może zostać zainfekowana. W kolejnych $n-1$ wierszach podane są trzy liczby całkowite $u_{i}$, $v_{i}$ i $a_{i}$ oznaczające identyfikatory dwóch połączonych serwerów $(1 \leq u_{i}, v_{i} \leq n)$ oraz 1, jeśli połączenie może zostać przerwane lub 0 w przeciwnym razie.

%WYJSCIE

    \section*{Wyjście}
    
    Na standardowe wyjście należy wypisać jedną liczbę całkowitą oznaczającą minimalną liczbę połączeń jakie Bitek musi odciąć. Jeżeli problem nie ma rozwiązania i Bitek nie jest w stanie odpowiednio zabezpieczyć serwerowni należy wypisać $-1$.

%PRZYKLAD

    \section*{Przykład}
    
    \noindent
    \begin{minipage}[t]{0.5\textwidth}
        Dla danych wejściowych:\vspace{1ex}\\
        \texttt{6 4\\3 5 1\\1 2 0\\2 3 1\\3 4 0\\5 6 1}
    \end{minipage}
    \begin{minipage}[t]{0.5\textwidth}
        poprawnym wynikiem jest:\vspace{1ex}\\
        \texttt{1}
    \end{minipage}
    
    \vspace{-10ex}
    \begin{center}
        \includegraphics{wirrys-1.pdf}
    \end{center}
    \noindent\textbf{Wyjaśnienie przykładu:} Bitek może odciąć połączenie pomiędzy serwerami numer 3 i 5 lub 2 i 3.
    
%OCENIANIE

    \section*{Ocenianie}
        
    Zestaw testów dzieli się na następujące podzadania. Testy do każdego podzadania składają się z jednej lub większej liczby osobnych grup testów.
    
    \begin{center}
        \begin{tabular}{ |c|p{9cm}|c| }
            \hline
            \textbf{Podzadanie} & \textbf{Warunki} & \textbf{Liczba punktów}\\
            \hline
            1 & $1 \leq m \leq n \leq 20$ & 20\\
            \hline
            2 & można odcinać tylko liście\footnotemark & 30\\
            \hline
            3 & brak dodatkowych ograniczeń & 50\\
            \hline
        \end{tabular}
        \footnotetext{liście - wierzchołki, z których wychodzi tylko jedna krawędź.}
    \end{center}

\end{document}
