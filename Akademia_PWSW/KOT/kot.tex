\documentclass[10pt]{article}
\usepackage[utf8x]{inputenc}
\usepackage{polski}

\usepackage{amsmath}
\usepackage{amsthm}
\usepackage{amssymb}

\usepackage{graphics}
\usepackage{pdfpages}
\DeclareGraphicsRule{.1}{mps}{*}{}
\usepackage{epstopdf}

\usepackage{geometry}
\geometry{a4paper, margin=2cm}

\begin{document}
    
%NAGLOWEK

    \noindent
    \begin{minipage}{0.5\textwidth}
        \LARGE{\textsf{\textbf{Zadanie: KOT\\Kot}}}
    \end{minipage}
    \begin{minipage}{0.5\textwidth}
        \begin{flushright}
            \includegraphics[height=1.5cm]{logo.jpg}
        \end{flushright}
    \end{minipage}
    
    \noindent\rule{\textwidth}{0.4pt}
    
    \noindent\textbf{Akademia Programowania PWSW, dzień I, Dostępna pamięć: 512 MB.}
    \vspace{1em}
    
%TRESC
    
    \noindent
    Jak powszechnie wiadomo, kiedy kota nie ma, myszy harcują, a harce te odbywają się w kuchni (kuchnia ma wymiary $n \times n$ i jest pokryta $n^{2}$ kwadratowymi płytkami), gdzie na każdej z płytek przesiaduje pewna liczba myszy (być może zero). Kot Bitom w czasie swojego polowania łapie myszy zajmujące pewne $k \times k$ płytek (nie mniej, nie więcej), a ponieważ myszy są szybkie, może on upolawać co najwyżej tyle myszy, ile znajduje się na najsłabiej okupowanej przez nie płytce.

    Napisz program, który dla każdego możliwego obszaru polowania, wyznaczy ile co najwyżej myszy może złapać kot Bitom.

%WEJSCIE

    \section*{Wejście}
    
    W pierwszym wierszu standardowego wejścia znajdują się dwie liczby całkowite $n$ i $k$, oznaczające odpowiednio wymiar kuchni i obszar polowania kota Bitoma. W $n$ kolejnych wierszach znajdują się ciągi $n$ liczb całkowitych z przedziału od 0 do $10^{9}$, oznaczające ile myszy znajduje się na każdej z płytek.

%WYJSCIE

    \section*{Wyjście}
    
    Na standardowe wyjście należy wypisać $n-k+1$ wierszy po $n-k+1$ liczb całkowitych w każdym, oznaczające ile co najwyżej myszy może upolować kot Bitom w danym rejonie kuchni.

%PRZYKLAD

    \section*{Przykład}
    
    \noindent
    \begin{minipage}[t]{0.5\textwidth}
        Dla danych wejściowych:\vspace{1ex}\\
        \texttt{4 2\\0 1 2 3\\4 5 6 7\\8 9 0 1\\2 3 4 0}
    \end{minipage}
    \begin{minipage}[t]{0.5\textwidth}
        poprawnym wynikiem jest:\vspace{1ex}\\
        \texttt{0 1 2\\4 0 0\\2 0 0}
    \end{minipage}

%OCENIANIE

    \section*{Ocenianie}
        
    Zestaw testów dzieli się na następujące podzadania. Testy do każdego podzadania składają się z jednej lub większej liczby osobnych grup testów.
    
    \begin{center}
        \begin{tabular}{ |c|p{9cm}|c| }
            \hline
            \textbf{Podzadanie} & \textbf{Warunki} & \textbf{Liczba punktów}\\
            \hline
            1 & $1 \leq k \leq n \leq 100$ & 10\\
            \hline
            2 & $1 \leq k \leq n \leq 600$ & 30\\
            \hline
            3 & $1 \leq k \leq n \leq 2000$ & 60\\
            \hline
        \end{tabular}
    \end{center}

\end{document}
\documentclass[10pt]{article}
\usepackage[utf8x]{inputenc}
\usepackage{polski}

