\documentclass[10pt]{article}
\usepackage[utf8x]{inputenc}
\usepackage{polski}

\usepackage{amsmath}
\usepackage{amsthm}
\usepackage{amssymb}

\usepackage{graphics}
\usepackage{pdfpages}
\DeclareGraphicsRule{.1}{mps}{*}{}
\usepackage{epstopdf}

% \newcommand{\hint}[2]{\begin{flushleft}\textbf{\ref{#1}}\quad #2\end{flushleft}}

\usepackage{geometry}
\geometry{a4paper, margin=2cm}

\begin{document}
    
%NAGLOWEK

    \noindent
    \begin{minipage}{0.5\textwidth}
        \LARGE{\textsf{\textbf{Zadanie: LNG\\Lampki - Nowa Generacja}}}
    \end{minipage}
    \begin{minipage}{0.5\textwidth}
        \begin{flushright}
            \includegraphics[height=1.5cm]{logo.png}
        \end{flushright}
    \end{minipage}
    
    \noindent\rule{\textwidth}{0.4pt}
    
    \noindent\textbf{Akademia Programowania PWSW, dzień ?, Dostępna pamięć: 128 MB.}
    \vspace{1em}
    
%TRESC
    
    \noindent
    W tym roku twórcy światełek choinkowych nie próżnowali i poza standardowymi światełkami, poszli z duchem czasu, kładąc nacisk na produkcję światełek w tęczowych zestawach kolorystycznych. Każdy dostępny zestaw \textit{Tęczowego blasku} składa się z ciągu $n$ lampek, każdej w jednym z $m$ kolorów (oznaczonych od 1 do $m$), oraz specjalnego przełącznika, który pozwala wyłączyć wszystkie lampki w kolorach z pewnego przedziału $[l, r]$, dla $1\leq l \leq r \leq m$. Twórcy nazywają przedział $[l, r]$ \textit{tęczowym}, jeżeli zapalone lampki tworzą tęczowy układ tj. każda kolejna zapalona lampka ma kolor nie mniejszy niż poprzednia zapalona, przeglądając łańcuch od lewej do prawej.
    
    Napisz program, który pomoże twórcom wyznaczyć liczbę tęczowych przedziałów dla danego zestawu światełek.
    
%WEJSCIE

    \section*{Wejście}
    
    W pierwszym wierszu standardowego wejścia znajdują się dwie liczby całkowite $n$ i $m$, oznaczające odpowiednio liczbę światełek w łańcuchu oraz liczbę kolorów. W drugim wierszu znajduje się ciąg $n$ liczb całkowitych z przedziału od 1 do $m$, oznaczających kolejne kolory lampek w danym zestawie światełek.

%WYJSCIE

    \section*{Wyjście}
    
    Na standardowe wyjście należy wypisać jedną liczbę całkowitą, oznaczającą liczbę wszystkich tęczowych przedziałów dla danego zestawu światełek.

%PRZYKLAD

    \section*{Przykład}
    
    \noindent
    \begin{minipage}[t]{0.5\textwidth}
        Dla danych wejściowych:\vspace{1ex}\\
        \texttt{3 3\\2 3 1}
    \end{minipage}
    \begin{minipage}[t]{0.5\textwidth}
        poprawnym wynikiem jest:\vspace{1ex}\\
        \texttt{4}
    \end{minipage}
    
    \vspace{2ex}
    \noindent\textbf{Wyjaśnienie przykładu:} Tęczowe przedziały to: [1, 1], [1, 2], [1, 3] i [2, 3]. 
    
    %\vspace{1.5em}
    %\noindent\textbf{Testy ,,ocen":}\vspace{.5ex}
    
    %\texttt{\textbf{1ocen:}} $n$ = 5, $m$ = 6; odpowiedź to 4;\vspace{.5ex}
    
%OCENIANIE

    \section*{Ocenianie}
        
    Zestaw testów dzieli się na następujące podzadania. Testy do każdego podzadania składają się z jednej lub większej liczby osobnych grup testów.
    
    \begin{center}
        \begin{tabular}{ |c|p{9cm}|c| }
            \hline
            \textbf{Podzadanie} & \textbf{Warunki} & \textbf{Liczba punktów}\\
            \hline
            1 & $1 \leq n, m \leq 100$ & 10\\
            \hline
            2 & $1 \leq n, m \leq 2000$ & 30\\
            \hline
            3 & $1 \leq n, m \leq 200000$ & 60\\
            \hline
            
        \end{tabular}
    \end{center}

\end{document}
