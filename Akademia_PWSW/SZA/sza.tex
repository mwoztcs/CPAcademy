\documentclass[10pt]{article}
\usepackage[utf8x]{inputenc}
\usepackage{polski}

\usepackage{amsmath}
\usepackage{amsthm}
\usepackage{amssymb}

\usepackage{graphics}
\usepackage{pdfpages}
\DeclareGraphicsRule{.1}{mps}{*}{}
\usepackage{epstopdf}

\usepackage{geometry}
\geometry{a4paper, margin=2cm}

\begin{document}
    
%NAGLOWEK

    \noindent
    \begin{minipage}{0.5\textwidth}
        \LARGE{\textsf{\textbf{Zadanie: SZA\\Szaliki}}}
    \end{minipage}
    \begin{minipage}{0.5\textwidth}
        \begin{flushright}
            \includegraphics[height=1.5cm]{logo.jpg}
        \end{flushright}
    \end{minipage}
    
    \noindent\rule{\textwidth}{0.4pt}
    
    \noindent\textbf{Akademia Programowania PWSW, dzień III, Dostępna pamięć: 128 MB.}
    \vspace{1em}
    
%TRESC
    
    \noindent
    Bajtosia dorabia wieczorami, robiąc szaliki na drutach. Dzisiaj ma szczęście - pojawił się kupiec zainteresowany jej wyrobami. Stawia jednak warunek - wszystkie szaliki jakie kupi mają być jednakowej długości (składać się z jednakowej liczby rzędów). Kupiec oznajmił, że ma do załatwienia jeszcza kilka spraw i wróci za $m$ chwil. Bajtosia wie, że w ciągu jednej chwili jest w stanie dorobić lub spruć jeden rząd oczek szalika.
    
    Znając długości poszczególnych szalików pomóż Bajtosi określić ile maksymalnie szalików będzie mogła sprzedać kupcowi, kiedy ten wróci.

%WEJSCIE

    \section*{Wejście}
    
    W pierwszym wierszu standardowego wejścia znajdują się dwie liczby całkowite $n$ i $m$ $(0\leq m\leq 10^{9})$, oznaczające liczbę szalików posiadanych przez Bajtosię oraz liczbę chwil, po których kupiec wróci do Bajtosi. W drugim wierszu znajduje się ciąg $n$ liczb całkowitych $r_{1}, r_{2}, \ldots, r_{n}$ $(1 \leq r_{i} \leq 10^{9})$, gdzie $r_{i}$ oznacza liczbę rzędów $i$-tego szalika.

%WYJSCIE

    \section*{Wyjście}
    
    Na standardowe wyjście należy wypisać jedną liczbę całkowitą, oznaczającą maksymalną liczbę szalików jakie Bajtosia będzie w stanie sprzedać kupcowi po niezbędnych przeróbkach.

%PRZYKLAD

    \section*{Przykład}
    
    \noindent
    \begin{minipage}[t]{0.5\textwidth}
        Dla danych wejściowych:\vspace{1ex}\\
        \texttt{5 6\\1 2 3 4 4}
    \end{minipage}
    \begin{minipage}[t]{0.5\textwidth}
        poprawnym wynikiem jest:\vspace{1ex}\\
        \texttt{5}
    \end{minipage}
    
    \vspace{2ex}
    \noindent\textbf{Wyjaśnienie przykładu:} Bajtosia w ciągu sześciu chwil jest w stanie przerobić wszystkie szaliki do długości 2.
    
%OCENIANIE

    \section*{Ocenianie}
        
    Zestaw testów dzieli się na następujące podzadania. Testy do każdego podzadania składają się z jednej lub większej liczby osobnych grup testów.
    
    \begin{center}
        \begin{tabular}{ |c|p{9cm}|c| }
            \hline
            \textbf{Podzadanie} & \textbf{Warunki} & \textbf{Liczba punktów}\\
            \hline
            1 & $1\leq n\leq 100$ & 20\\
            \hline
            2 & $1\leq n\leq 3000$ & 30\\
            \hline
            3 & $1\leq n\leq 200000$ & 50\\
            \hline
        \end{tabular}
    \end{center}

\end{document}
