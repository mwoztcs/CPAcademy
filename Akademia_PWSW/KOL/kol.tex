\documentclass[10pt]{article}
\usepackage[utf8x]{inputenc}
\usepackage{polski}

\usepackage{amsmath}
\usepackage{amsthm}
\usepackage{amssymb}

\usepackage{graphics}
\usepackage{pdfpages}
\DeclareGraphicsRule{.1}{mps}{*}{}
\usepackage{epstopdf}

% \newcommand{\hint}[2]{\begin{flushleft}\textbf{\ref{#1}}\quad #2\end{flushleft}}

\usepackage{geometry}
\geometry{a4paper, margin=2cm}

\begin{document}
    
%NAGLOWEK

    \noindent
    \begin{minipage}{0.5\textwidth}
        \LARGE{\textsf{\textbf{Zadanie: KOL\\Kolacja przy świecach}}}
    \end{minipage}
    \begin{minipage}{0.5\textwidth}
        \begin{flushright}
            \includegraphics[height=1.5cm]{logo.png}
        \end{flushright}
    \end{minipage}
    
    \noindent\rule{\textwidth}{0.4pt}
    
    \noindent\textbf{Akademia Programowania PWSW, dzień ?, Dostępna pamięć: 128 MB.}
    \vspace{1em}
    
%TRESC
    
    \noindent
    Właściciel jednej z najznakomitszych restauracji w całej Bajtocji musi zadbać o każdy szczegół, aby jego goście byli zadowoleni. W ciągu najbliższych $m$ dni w restauracji odbędą się romantyczne kolacje przy świecach. Dla każdej rezerwacji właściciel wie, ile świec będzie musiał zapalić, a dzięki swojemu długoletniemu doświadczeniu wie, że każda świeca w czasie takiej kolacji spala się o jedną jednostkę wysokości.
    
    Znając zestaw świec jakie posiada właściciel i wymagania gości odnośnie każdej kolacji określ na ile wieczorów wystarczy posiadany przez właściciela zestaw.

%WEJSCIE

    \section*{Wejście}
    
    W pierwszym wierszu standardowego wejścia znajdują się dwie liczby całkowite $n$ i $m$ $(1\leq n, m\leq 10^{5})$, oznaczające odpowiednio liczbę świec i liczbę wieczorów. W kolejnym wierszu znajduje się ciąg $n$ liczb całkowitych $h_{i}$ $(1\leq h_{i}\leq 10^{5})$, oznaczający wysokości poszczególnych świec. W następnym wierszu znajduje się ciąg $m$ liczb całkowitych $c_{j}$ $(1\leq c_{j}\leq 10^{5})$, oznaczający liczbę świec potrzebnych na $j$-tą kolację.

%WYJSCIE

    \section*{Wyjście}
    
    Na standardowe wyjście należy wypisać jedną liczbę całkowitą, oznaczającą liczbę wieczorów na które wystarczy posiadany przez właściciela zestaw świec.

%PRZYKLAD

    \section*{Przykład}
    
    \noindent
    \begin{minipage}[t]{0.5\textwidth}
        Dla danych wejściowych:\vspace{1ex}\\
        \texttt{3 5\\1 2 5\\1 2 3 2 1}
    \end{minipage}
    \begin{minipage}[t]{0.5\textwidth}
        poprawnym wynikiem jest:\vspace{1ex}\\
        \texttt{3}
    \end{minipage}
    
    \vspace{2ex}
    \noindent\textbf{Wyjaśnienie przykładu:} Niezależnie od tego które świece właścieciel będzie zapalał na kolejne kolacje, nie będzie posiadał dwóch świec na czwartą kolację. Wobec tego świec wystarczy jedynie na trzy pierwsze wieczory.
    
    
%OCENIANIE

    \section*{Ocenianie}
        
    Zestaw testów dzieli się na następujące podzadania. Testy do każdego podzadania składają się z jednej lub większej liczby osobnych grup testów.
    
    \begin{center}
        \begin{tabular}{ |c|p{9cm}|c| }
            \hline
            \textbf{Podzadanie} & \textbf{Warunki} & \textbf{Liczba punktów}\\
            \hline
            1 & $1 \leq n, m \leq 1000$ & 25\\
            \hline
            2 & wszystkie świeczki mają co najwyżej 10 różnych wysokości & 30\\
            \hline
            3 & brak dodatkowych ograniczeń & 45\\
            \hline
        \end{tabular}
    \end{center}

\end{document}
\documentclass[10pt]{article}
\usepackage[utf8x]{inputenc}
\usepackage{polski}

