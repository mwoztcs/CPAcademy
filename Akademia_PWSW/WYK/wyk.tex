\documentclass[10pt]{article}
\usepackage[utf8x]{inputenc}
\usepackage{polski}

\usepackage{amsmath}
\usepackage{amsthm}
\usepackage{amssymb}

\usepackage{graphics}
\usepackage{pdfpages}
\DeclareGraphicsRule{.1}{mps}{*}{}
\usepackage{epstopdf}

\usepackage{geometry}
\geometry{a4paper, margin=2cm}

\begin{document}
    
%NAGLOWEK

    \noindent
    \begin{minipage}{0.5\textwidth}
        \LARGE{\textsf{\textbf{Zadanie: WYK\\Wykopaliska}}}
    \end{minipage}
    \begin{minipage}{0.5\textwidth}
        \begin{flushright}
            \includegraphics[height=1.5cm]{logo.jpg}
        \end{flushright}
    \end{minipage}
    
    \noindent\rule{\textwidth}{0.4pt}
    
    \noindent\textbf{Akademia Programowania PWSW, dzień IV, Dostępna pamięć: 128 MB.}
    \vspace{1em}
    
%TRESC
    
    \noindent
    Nadworny archeolog króla Bajtazara prowadzi wykopaliska w Bajrucie. Obszar wykopalisk jest prostokątem, który dla uproszczenia prac archeologicznych został podzielony na $n$ rzędów i $m$ kolumn. Nadworny archeolog zlecił swoim podwładnym zestaw prac do wykonania, a teraz musi złożyć przed królem sprawozdanie z postępu prac.
    
    Pomóż nadwornemu archeologowi i napisz program, który dla zleconego zestawu prac wyznaczy głębokość na jaką zeszli archeologowie w każdym z sektorów.

%WEJSCIE

    \section*{Wejście}
    
    W pierwszym wierszu standardowego wejścia znajdują się trzy liczby całkowite $n$, $m$ $(1\leq n, m\leq 200)$ i $q$ $(1\leq q\leq 200000)$, oznaczające odpowiednio liczbę rzędów (numerowane od 1 do $n$, z góry na dół), liczbę kolumn (numerowane od 1 do $m$, od lewej do prawej) i liczbę zleceń jakie wykonali archeolodzy.

    W $q$ kolejnych wierszach znajdują się opisy zleconych prac. Możliwe są cztery rodzaje prac podane w następujących formatach:

    \texttt{S} $r$ $c$ - pogłębienie pojedynczego sektora na przecięciu rzędu $r$ i kolumny $c$ o jedną jednostkę. 

    \texttt{R} $r$ - wyrównanie wszystkich sektorów w rzędzie $r$ do głębokości najgłębszego z sektorów w tym rzędzie. 

    \texttt{C} $c$ - wyrównanie wszystkich sektorów w kolumnie $c$ do głębokości najgłębszego z sektorów w tej kolumnie. 

    \texttt{A} - wyrównanie wszystkich sektorów do głębokości najgłębszego z sektorów na całym obszarze wykopalisk.

%WYJSCIE

    \section*{Wyjście}
    
    Na standardowe wyjście należy wypisać $n$ wierszy po $m$ liczb całkowitych w każdym, oznaczające na jaką głębokość archeolodzy zeszli w każdym sektorze.

%PRZYKLAD

    \section*{Przykład}
    
    \noindent
    \begin{minipage}[t]{0.5\textwidth}
        Dla danych wejściowych:\vspace{1ex}\\
        \texttt{3 4 5\\S 1 1\\A\\S 2 3\\C 3\\R 3}
    \end{minipage}
    \begin{minipage}[t]{0.5\textwidth}
        poprawnym wynikiem jest:\vspace{1ex}\\
        \texttt{1 1 2 1\\1 1 2 1\\2 2 2 2}
    \end{minipage}
    
%OCENIANIE

    \section*{Ocenianie}
        
    Zestaw testów dzieli się na następujące podzadania. Testy do każdego podzadania składają się z jednej lub większej liczby osobnych grup testów.
    
    \begin{center}
        \begin{tabular}{ |c|p{9cm}|c| }
            \hline
            \textbf{Podzadanie} & \textbf{Warunki} & \textbf{Liczba punktów}\\
            \hline
            1 & $1 \leq q \leq 1000$ & 20\\
            \hline
            2 & brak prac typu \texttt{R} i \texttt{C} & 20\\
            \hline
            3 & brak prac typu \texttt{C} & 20\\
            \hline
            4 & brak dodatkowych ograniczeń & 40\\
            \hline
        \end{tabular}
    \end{center}

\end{document}
